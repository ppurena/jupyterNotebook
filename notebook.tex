
% Default to the notebook output style

    


% Inherit from the specified cell style.




    
\documentclass[11pt]{article}

    
    
    \usepackage[T1]{fontenc}
    % Nicer default font (+ math font) than Computer Modern for most use cases
    \usepackage{mathpazo}

    % Basic figure setup, for now with no caption control since it's done
    % automatically by Pandoc (which extracts ![](path) syntax from Markdown).
    \usepackage{graphicx}
    % We will generate all images so they have a width \maxwidth. This means
    % that they will get their normal width if they fit onto the page, but
    % are scaled down if they would overflow the margins.
    \makeatletter
    \def\maxwidth{\ifdim\Gin@nat@width>\linewidth\linewidth
    \else\Gin@nat@width\fi}
    \makeatother
    \let\Oldincludegraphics\includegraphics
    % Set max figure width to be 80% of text width, for now hardcoded.
    \renewcommand{\includegraphics}[1]{\Oldincludegraphics[width=.8\maxwidth]{#1}}
    % Ensure that by default, figures have no caption (until we provide a
    % proper Figure object with a Caption API and a way to capture that
    % in the conversion process - todo).
    \usepackage{caption}
    \DeclareCaptionLabelFormat{nolabel}{}
    \captionsetup{labelformat=nolabel}

    \usepackage{adjustbox} % Used to constrain images to a maximum size 
    \usepackage{xcolor} % Allow colors to be defined
    \usepackage{enumerate} % Needed for markdown enumerations to work
    \usepackage{geometry} % Used to adjust the document margins
    \usepackage{amsmath} % Equations
    \usepackage{amssymb} % Equations
    \usepackage{textcomp} % defines textquotesingle
    % Hack from http://tex.stackexchange.com/a/47451/13684:
    \AtBeginDocument{%
        \def\PYZsq{\textquotesingle}% Upright quotes in Pygmentized code
    }
    \usepackage{upquote} % Upright quotes for verbatim code
    \usepackage{eurosym} % defines \euro
    \usepackage[mathletters]{ucs} % Extended unicode (utf-8) support
    \usepackage[utf8x]{inputenc} % Allow utf-8 characters in the tex document
    \usepackage{fancyvrb} % verbatim replacement that allows latex
    \usepackage{grffile} % extends the file name processing of package graphics 
                         % to support a larger range 
    % The hyperref package gives us a pdf with properly built
    % internal navigation ('pdf bookmarks' for the table of contents,
    % internal cross-reference links, web links for URLs, etc.)
    \usepackage{hyperref}
    \usepackage{longtable} % longtable support required by pandoc >1.10
    \usepackage{booktabs}  % table support for pandoc > 1.12.2
    \usepackage[inline]{enumitem} % IRkernel/repr support (it uses the enumerate* environment)
    \usepackage[normalem]{ulem} % ulem is needed to support strikethroughs (\sout)
                                % normalem makes italics be italics, not underlines
    

    
    
    % Colors for the hyperref package
    \definecolor{urlcolor}{rgb}{0,.145,.698}
    \definecolor{linkcolor}{rgb}{.71,0.21,0.01}
    \definecolor{citecolor}{rgb}{.12,.54,.11}

    % ANSI colors
    \definecolor{ansi-black}{HTML}{3E424D}
    \definecolor{ansi-black-intense}{HTML}{282C36}
    \definecolor{ansi-red}{HTML}{E75C58}
    \definecolor{ansi-red-intense}{HTML}{B22B31}
    \definecolor{ansi-green}{HTML}{00A250}
    \definecolor{ansi-green-intense}{HTML}{007427}
    \definecolor{ansi-yellow}{HTML}{DDB62B}
    \definecolor{ansi-yellow-intense}{HTML}{B27D12}
    \definecolor{ansi-blue}{HTML}{208FFB}
    \definecolor{ansi-blue-intense}{HTML}{0065CA}
    \definecolor{ansi-magenta}{HTML}{D160C4}
    \definecolor{ansi-magenta-intense}{HTML}{A03196}
    \definecolor{ansi-cyan}{HTML}{60C6C8}
    \definecolor{ansi-cyan-intense}{HTML}{258F8F}
    \definecolor{ansi-white}{HTML}{C5C1B4}
    \definecolor{ansi-white-intense}{HTML}{A1A6B2}

    % commands and environments needed by pandoc snippets
    % extracted from the output of `pandoc -s`
    \providecommand{\tightlist}{%
      \setlength{\itemsep}{0pt}\setlength{\parskip}{0pt}}
    \DefineVerbatimEnvironment{Highlighting}{Verbatim}{commandchars=\\\{\}}
    % Add ',fontsize=\small' for more characters per line
    \newenvironment{Shaded}{}{}
    \newcommand{\KeywordTok}[1]{\textcolor[rgb]{0.00,0.44,0.13}{\textbf{{#1}}}}
    \newcommand{\DataTypeTok}[1]{\textcolor[rgb]{0.56,0.13,0.00}{{#1}}}
    \newcommand{\DecValTok}[1]{\textcolor[rgb]{0.25,0.63,0.44}{{#1}}}
    \newcommand{\BaseNTok}[1]{\textcolor[rgb]{0.25,0.63,0.44}{{#1}}}
    \newcommand{\FloatTok}[1]{\textcolor[rgb]{0.25,0.63,0.44}{{#1}}}
    \newcommand{\CharTok}[1]{\textcolor[rgb]{0.25,0.44,0.63}{{#1}}}
    \newcommand{\StringTok}[1]{\textcolor[rgb]{0.25,0.44,0.63}{{#1}}}
    \newcommand{\CommentTok}[1]{\textcolor[rgb]{0.38,0.63,0.69}{\textit{{#1}}}}
    \newcommand{\OtherTok}[1]{\textcolor[rgb]{0.00,0.44,0.13}{{#1}}}
    \newcommand{\AlertTok}[1]{\textcolor[rgb]{1.00,0.00,0.00}{\textbf{{#1}}}}
    \newcommand{\FunctionTok}[1]{\textcolor[rgb]{0.02,0.16,0.49}{{#1}}}
    \newcommand{\RegionMarkerTok}[1]{{#1}}
    \newcommand{\ErrorTok}[1]{\textcolor[rgb]{1.00,0.00,0.00}{\textbf{{#1}}}}
    \newcommand{\NormalTok}[1]{{#1}}
    
    % Additional commands for more recent versions of Pandoc
    \newcommand{\ConstantTok}[1]{\textcolor[rgb]{0.53,0.00,0.00}{{#1}}}
    \newcommand{\SpecialCharTok}[1]{\textcolor[rgb]{0.25,0.44,0.63}{{#1}}}
    \newcommand{\VerbatimStringTok}[1]{\textcolor[rgb]{0.25,0.44,0.63}{{#1}}}
    \newcommand{\SpecialStringTok}[1]{\textcolor[rgb]{0.73,0.40,0.53}{{#1}}}
    \newcommand{\ImportTok}[1]{{#1}}
    \newcommand{\DocumentationTok}[1]{\textcolor[rgb]{0.73,0.13,0.13}{\textit{{#1}}}}
    \newcommand{\AnnotationTok}[1]{\textcolor[rgb]{0.38,0.63,0.69}{\textbf{\textit{{#1}}}}}
    \newcommand{\CommentVarTok}[1]{\textcolor[rgb]{0.38,0.63,0.69}{\textbf{\textit{{#1}}}}}
    \newcommand{\VariableTok}[1]{\textcolor[rgb]{0.10,0.09,0.49}{{#1}}}
    \newcommand{\ControlFlowTok}[1]{\textcolor[rgb]{0.00,0.44,0.13}{\textbf{{#1}}}}
    \newcommand{\OperatorTok}[1]{\textcolor[rgb]{0.40,0.40,0.40}{{#1}}}
    \newcommand{\BuiltInTok}[1]{{#1}}
    \newcommand{\ExtensionTok}[1]{{#1}}
    \newcommand{\PreprocessorTok}[1]{\textcolor[rgb]{0.74,0.48,0.00}{{#1}}}
    \newcommand{\AttributeTok}[1]{\textcolor[rgb]{0.49,0.56,0.16}{{#1}}}
    \newcommand{\InformationTok}[1]{\textcolor[rgb]{0.38,0.63,0.69}{\textbf{\textit{{#1}}}}}
    \newcommand{\WarningTok}[1]{\textcolor[rgb]{0.38,0.63,0.69}{\textbf{\textit{{#1}}}}}
    
    
    % Define a nice break command that doesn't care if a line doesn't already
    % exist.
    \def\br{\hspace*{\fill} \\* }
    % Math Jax compatability definitions
    \def\gt{>}
    \def\lt{<}
    % Document parameters
    \title{Curso Conieem Tecnol?gico de M?rida}
    
    
    

    % Pygments definitions
    
\makeatletter
\def\PY@reset{\let\PY@it=\relax \let\PY@bf=\relax%
    \let\PY@ul=\relax \let\PY@tc=\relax%
    \let\PY@bc=\relax \let\PY@ff=\relax}
\def\PY@tok#1{\csname PY@tok@#1\endcsname}
\def\PY@toks#1+{\ifx\relax#1\empty\else%
    \PY@tok{#1}\expandafter\PY@toks\fi}
\def\PY@do#1{\PY@bc{\PY@tc{\PY@ul{%
    \PY@it{\PY@bf{\PY@ff{#1}}}}}}}
\def\PY#1#2{\PY@reset\PY@toks#1+\relax+\PY@do{#2}}

\expandafter\def\csname PY@tok@w\endcsname{\def\PY@tc##1{\textcolor[rgb]{0.73,0.73,0.73}{##1}}}
\expandafter\def\csname PY@tok@c\endcsname{\let\PY@it=\textit\def\PY@tc##1{\textcolor[rgb]{0.25,0.50,0.50}{##1}}}
\expandafter\def\csname PY@tok@cp\endcsname{\def\PY@tc##1{\textcolor[rgb]{0.74,0.48,0.00}{##1}}}
\expandafter\def\csname PY@tok@k\endcsname{\let\PY@bf=\textbf\def\PY@tc##1{\textcolor[rgb]{0.00,0.50,0.00}{##1}}}
\expandafter\def\csname PY@tok@kp\endcsname{\def\PY@tc##1{\textcolor[rgb]{0.00,0.50,0.00}{##1}}}
\expandafter\def\csname PY@tok@kt\endcsname{\def\PY@tc##1{\textcolor[rgb]{0.69,0.00,0.25}{##1}}}
\expandafter\def\csname PY@tok@o\endcsname{\def\PY@tc##1{\textcolor[rgb]{0.40,0.40,0.40}{##1}}}
\expandafter\def\csname PY@tok@ow\endcsname{\let\PY@bf=\textbf\def\PY@tc##1{\textcolor[rgb]{0.67,0.13,1.00}{##1}}}
\expandafter\def\csname PY@tok@nb\endcsname{\def\PY@tc##1{\textcolor[rgb]{0.00,0.50,0.00}{##1}}}
\expandafter\def\csname PY@tok@nf\endcsname{\def\PY@tc##1{\textcolor[rgb]{0.00,0.00,1.00}{##1}}}
\expandafter\def\csname PY@tok@nc\endcsname{\let\PY@bf=\textbf\def\PY@tc##1{\textcolor[rgb]{0.00,0.00,1.00}{##1}}}
\expandafter\def\csname PY@tok@nn\endcsname{\let\PY@bf=\textbf\def\PY@tc##1{\textcolor[rgb]{0.00,0.00,1.00}{##1}}}
\expandafter\def\csname PY@tok@ne\endcsname{\let\PY@bf=\textbf\def\PY@tc##1{\textcolor[rgb]{0.82,0.25,0.23}{##1}}}
\expandafter\def\csname PY@tok@nv\endcsname{\def\PY@tc##1{\textcolor[rgb]{0.10,0.09,0.49}{##1}}}
\expandafter\def\csname PY@tok@no\endcsname{\def\PY@tc##1{\textcolor[rgb]{0.53,0.00,0.00}{##1}}}
\expandafter\def\csname PY@tok@nl\endcsname{\def\PY@tc##1{\textcolor[rgb]{0.63,0.63,0.00}{##1}}}
\expandafter\def\csname PY@tok@ni\endcsname{\let\PY@bf=\textbf\def\PY@tc##1{\textcolor[rgb]{0.60,0.60,0.60}{##1}}}
\expandafter\def\csname PY@tok@na\endcsname{\def\PY@tc##1{\textcolor[rgb]{0.49,0.56,0.16}{##1}}}
\expandafter\def\csname PY@tok@nt\endcsname{\let\PY@bf=\textbf\def\PY@tc##1{\textcolor[rgb]{0.00,0.50,0.00}{##1}}}
\expandafter\def\csname PY@tok@nd\endcsname{\def\PY@tc##1{\textcolor[rgb]{0.67,0.13,1.00}{##1}}}
\expandafter\def\csname PY@tok@s\endcsname{\def\PY@tc##1{\textcolor[rgb]{0.73,0.13,0.13}{##1}}}
\expandafter\def\csname PY@tok@sd\endcsname{\let\PY@it=\textit\def\PY@tc##1{\textcolor[rgb]{0.73,0.13,0.13}{##1}}}
\expandafter\def\csname PY@tok@si\endcsname{\let\PY@bf=\textbf\def\PY@tc##1{\textcolor[rgb]{0.73,0.40,0.53}{##1}}}
\expandafter\def\csname PY@tok@se\endcsname{\let\PY@bf=\textbf\def\PY@tc##1{\textcolor[rgb]{0.73,0.40,0.13}{##1}}}
\expandafter\def\csname PY@tok@sr\endcsname{\def\PY@tc##1{\textcolor[rgb]{0.73,0.40,0.53}{##1}}}
\expandafter\def\csname PY@tok@ss\endcsname{\def\PY@tc##1{\textcolor[rgb]{0.10,0.09,0.49}{##1}}}
\expandafter\def\csname PY@tok@sx\endcsname{\def\PY@tc##1{\textcolor[rgb]{0.00,0.50,0.00}{##1}}}
\expandafter\def\csname PY@tok@m\endcsname{\def\PY@tc##1{\textcolor[rgb]{0.40,0.40,0.40}{##1}}}
\expandafter\def\csname PY@tok@gh\endcsname{\let\PY@bf=\textbf\def\PY@tc##1{\textcolor[rgb]{0.00,0.00,0.50}{##1}}}
\expandafter\def\csname PY@tok@gu\endcsname{\let\PY@bf=\textbf\def\PY@tc##1{\textcolor[rgb]{0.50,0.00,0.50}{##1}}}
\expandafter\def\csname PY@tok@gd\endcsname{\def\PY@tc##1{\textcolor[rgb]{0.63,0.00,0.00}{##1}}}
\expandafter\def\csname PY@tok@gi\endcsname{\def\PY@tc##1{\textcolor[rgb]{0.00,0.63,0.00}{##1}}}
\expandafter\def\csname PY@tok@gr\endcsname{\def\PY@tc##1{\textcolor[rgb]{1.00,0.00,0.00}{##1}}}
\expandafter\def\csname PY@tok@ge\endcsname{\let\PY@it=\textit}
\expandafter\def\csname PY@tok@gs\endcsname{\let\PY@bf=\textbf}
\expandafter\def\csname PY@tok@gp\endcsname{\let\PY@bf=\textbf\def\PY@tc##1{\textcolor[rgb]{0.00,0.00,0.50}{##1}}}
\expandafter\def\csname PY@tok@go\endcsname{\def\PY@tc##1{\textcolor[rgb]{0.53,0.53,0.53}{##1}}}
\expandafter\def\csname PY@tok@gt\endcsname{\def\PY@tc##1{\textcolor[rgb]{0.00,0.27,0.87}{##1}}}
\expandafter\def\csname PY@tok@err\endcsname{\def\PY@bc##1{\setlength{\fboxsep}{0pt}\fcolorbox[rgb]{1.00,0.00,0.00}{1,1,1}{\strut ##1}}}
\expandafter\def\csname PY@tok@kc\endcsname{\let\PY@bf=\textbf\def\PY@tc##1{\textcolor[rgb]{0.00,0.50,0.00}{##1}}}
\expandafter\def\csname PY@tok@kd\endcsname{\let\PY@bf=\textbf\def\PY@tc##1{\textcolor[rgb]{0.00,0.50,0.00}{##1}}}
\expandafter\def\csname PY@tok@kn\endcsname{\let\PY@bf=\textbf\def\PY@tc##1{\textcolor[rgb]{0.00,0.50,0.00}{##1}}}
\expandafter\def\csname PY@tok@kr\endcsname{\let\PY@bf=\textbf\def\PY@tc##1{\textcolor[rgb]{0.00,0.50,0.00}{##1}}}
\expandafter\def\csname PY@tok@bp\endcsname{\def\PY@tc##1{\textcolor[rgb]{0.00,0.50,0.00}{##1}}}
\expandafter\def\csname PY@tok@fm\endcsname{\def\PY@tc##1{\textcolor[rgb]{0.00,0.00,1.00}{##1}}}
\expandafter\def\csname PY@tok@vc\endcsname{\def\PY@tc##1{\textcolor[rgb]{0.10,0.09,0.49}{##1}}}
\expandafter\def\csname PY@tok@vg\endcsname{\def\PY@tc##1{\textcolor[rgb]{0.10,0.09,0.49}{##1}}}
\expandafter\def\csname PY@tok@vi\endcsname{\def\PY@tc##1{\textcolor[rgb]{0.10,0.09,0.49}{##1}}}
\expandafter\def\csname PY@tok@vm\endcsname{\def\PY@tc##1{\textcolor[rgb]{0.10,0.09,0.49}{##1}}}
\expandafter\def\csname PY@tok@sa\endcsname{\def\PY@tc##1{\textcolor[rgb]{0.73,0.13,0.13}{##1}}}
\expandafter\def\csname PY@tok@sb\endcsname{\def\PY@tc##1{\textcolor[rgb]{0.73,0.13,0.13}{##1}}}
\expandafter\def\csname PY@tok@sc\endcsname{\def\PY@tc##1{\textcolor[rgb]{0.73,0.13,0.13}{##1}}}
\expandafter\def\csname PY@tok@dl\endcsname{\def\PY@tc##1{\textcolor[rgb]{0.73,0.13,0.13}{##1}}}
\expandafter\def\csname PY@tok@s2\endcsname{\def\PY@tc##1{\textcolor[rgb]{0.73,0.13,0.13}{##1}}}
\expandafter\def\csname PY@tok@sh\endcsname{\def\PY@tc##1{\textcolor[rgb]{0.73,0.13,0.13}{##1}}}
\expandafter\def\csname PY@tok@s1\endcsname{\def\PY@tc##1{\textcolor[rgb]{0.73,0.13,0.13}{##1}}}
\expandafter\def\csname PY@tok@mb\endcsname{\def\PY@tc##1{\textcolor[rgb]{0.40,0.40,0.40}{##1}}}
\expandafter\def\csname PY@tok@mf\endcsname{\def\PY@tc##1{\textcolor[rgb]{0.40,0.40,0.40}{##1}}}
\expandafter\def\csname PY@tok@mh\endcsname{\def\PY@tc##1{\textcolor[rgb]{0.40,0.40,0.40}{##1}}}
\expandafter\def\csname PY@tok@mi\endcsname{\def\PY@tc##1{\textcolor[rgb]{0.40,0.40,0.40}{##1}}}
\expandafter\def\csname PY@tok@il\endcsname{\def\PY@tc##1{\textcolor[rgb]{0.40,0.40,0.40}{##1}}}
\expandafter\def\csname PY@tok@mo\endcsname{\def\PY@tc##1{\textcolor[rgb]{0.40,0.40,0.40}{##1}}}
\expandafter\def\csname PY@tok@ch\endcsname{\let\PY@it=\textit\def\PY@tc##1{\textcolor[rgb]{0.25,0.50,0.50}{##1}}}
\expandafter\def\csname PY@tok@cm\endcsname{\let\PY@it=\textit\def\PY@tc##1{\textcolor[rgb]{0.25,0.50,0.50}{##1}}}
\expandafter\def\csname PY@tok@cpf\endcsname{\let\PY@it=\textit\def\PY@tc##1{\textcolor[rgb]{0.25,0.50,0.50}{##1}}}
\expandafter\def\csname PY@tok@c1\endcsname{\let\PY@it=\textit\def\PY@tc##1{\textcolor[rgb]{0.25,0.50,0.50}{##1}}}
\expandafter\def\csname PY@tok@cs\endcsname{\let\PY@it=\textit\def\PY@tc##1{\textcolor[rgb]{0.25,0.50,0.50}{##1}}}

\def\PYZbs{\char`\\}
\def\PYZus{\char`\_}
\def\PYZob{\char`\{}
\def\PYZcb{\char`\}}
\def\PYZca{\char`\^}
\def\PYZam{\char`\&}
\def\PYZlt{\char`\<}
\def\PYZgt{\char`\>}
\def\PYZsh{\char`\#}
\def\PYZpc{\char`\%}
\def\PYZdl{\char`\$}
\def\PYZhy{\char`\-}
\def\PYZsq{\char`\'}
\def\PYZdq{\char`\"}
\def\PYZti{\char`\~}
% for compatibility with earlier versions
\def\PYZat{@}
\def\PYZlb{[}
\def\PYZrb{]}
\makeatother


    % Exact colors from NB
    \definecolor{incolor}{rgb}{0.0, 0.0, 0.5}
    \definecolor{outcolor}{rgb}{0.545, 0.0, 0.0}



    
    % Prevent overflowing lines due to hard-to-break entities
    \sloppy 
    % Setup hyperref package
    \hypersetup{
      breaklinks=true,  % so long urls are correctly broken across lines
      colorlinks=true,
      urlcolor=urlcolor,
      linkcolor=linkcolor,
      citecolor=citecolor,
      }
    % Slightly bigger margins than the latex defaults
    
    \geometry{verbose,tmargin=1in,bmargin=1in,lmargin=1in,rmargin=1in}
    
    

    \begin{document}
    
    
    \maketitle
    
    

    
    \hypertarget{introducciuxf3n-a-jupyter-notebooks-con-python}{%
\section{Introducción a Jupyter Notebooks con
Python}\label{introducciuxf3n-a-jupyter-notebooks-con-python}}

\hypertarget{ing.-josuxe9-alfonso-ureuxf1a-pajuxf3n}{%
\subsubsection{Ing. José Alfonso Ureña
Pajón}\label{ing.-josuxe9-alfonso-ureuxf1a-pajuxf3n}}

\hypertarget{egresado-de-la-carrera-de-ingenieruxeda-electruxf3nica-del-tecnoluxf3gico-de-muxe9rida}{%
\subparagraph{Egresado de la carrera de Ingeniería Electrónica del
Tecnológico de
Mérida}\label{egresado-de-la-carrera-de-ingenieruxeda-electruxf3nica-del-tecnoluxf3gico-de-muxe9rida}}

    \hypertarget{instalaciuxf3n}{%
\section{Instalación}\label{instalaciuxf3n}}

La forma más fácil de instalar la aplicación Jupyter Notebook es
instalar una distribución científica de Python llamada
\href{https://www.anaconda.com/download/\#windows}{Anaconda} que sirve
para analisis de datos y que incluye una versión de Python y todas las
librerias necesarias. (Se recomienda instalarlo usando la
configuraciones predeterminadas)

    \hypertarget{contenido-del-taller}{%
\section{Contenido del taller}\label{contenido-del-taller}}

\hypertarget{descripcion-de-el-uso-de-los-notebooks.}{%
\subsubsection{1. Descripcion de el uso de los
notebooks.}\label{descripcion-de-el-uso-de-los-notebooks.}}

\begin{itemize}
\tightlist
\item
  Descripcion y uso de cada una de las partes.
\item
  Ejericicios
\end{itemize}

\hypertarget{introducciuxf3n-a-python.}{%
\subsubsection{2. Introducción a
Python.}\label{introducciuxf3n-a-python.}}

\begin{itemize}
\tightlist
\item
  Variables, Sentencias de Control, Listas, Funciones
\item
  Ejercicios
\end{itemize}

\hypertarget{vistazo-a-topicos-avanzados.}{%
\subsubsection{3. Vistazo a Topicos
Avanzados.}\label{vistazo-a-topicos-avanzados.}}

\begin{itemize}
\tightlist
\item
  Matplotlib, Numpy, Pandas
\end{itemize}

    \hypertarget{quuxe9-es-jupyter-notebook}{%
\section{¿Qué es Jupyter Notebook?}\label{quuxe9-es-jupyter-notebook}}

\textbf{Jupyter Notebook} es un entorno de programación interactivo
creado por la comunidad que permite la elaboración de documentos
``notebook'' que pueden incluir: * Código ejecutable * Widgets
interactivos * Gráficas * Texto * Ecuaciones * Imágenes * Video

\textbf{Jupyter Notebook} amplía el uso de la terminal para programación
interactiva, dandole un giro nuevo, por medio de una aplicación web
diseñada para capturar todo el proceso de desarrollo de programa:
desarrollo, documentación, ejecución, así como reporte de resultados.

    \hypertarget{componentes}{%
\section{Componentes}\label{componentes}}

\textbf{Jupyter Notebook} combina tres componentes:

\begin{itemize}
\tightlist
\item
  \textbf{La Aplicación Web de Jupyter Notebook}: una aplicación web
  interactiva donde se crean los documentos ``notebook''.
\item
  \textbf{Kernels}: procesos separados iniciados por la aplicación web
  de notebook que ejecuta el código de los usuarios en un lenguaje de
  programación determinado y devuelve la salida a la aplicación web de
  notebook.
\item
  \textbf{Documentos Notebook}: archivo con extensión \emph{ipynb} que
  es la representación de todo el contenido visible en la aplicación web
  del notebook, incluidas las entradas y salidas de los cálculos, el
  texto, las ecuaciones, las imágenes y las representaciones multimedia
  de objetos. \emph{Cada documento Notebook tiene su propio kernel}.
\end{itemize}

    \hypertarget{para-que-pueden-servir-los-notebooks}{%
\section{¿Para que pueden servir los
Notebooks?}\label{para-que-pueden-servir-los-notebooks}}

\begin{itemize}
\tightlist
\item
  Elaboración de tareas
\item
  Reportes
\item
  Presentaciónes
\item
  Cursos de todo tipo (no sólo de programación)
\item
  Clases
\item
  Trabajo Científico
\item
  Analisis de datos
\item
  etc\ldots{}
\end{itemize}

    \hypertarget{inicio-del-servidor-de-notebook}{%
\section{Inicio del Servidor de
Notebook}\label{inicio-del-servidor-de-notebook}}

Se puede iniciar el ``Servidor de Notebook'' desde la línea de comandos
con el comando:

\emph{jupyter notebook}

    \hypertarget{panel-de-notebook-dashboard}{%
\section{Panel de Notebook
(Dashboard)}\label{panel-de-notebook-dashboard}}

Despues de iniciar el servidor de notebook, el navegador abrirá el Panel
de Notebook o Dashboard. Este es la página de inicio del Jupyter
Notebook y su propósito principal es mostrar los notebooks y archivos en
el directorio actual.

    \hypertarget{nuevo-notebook}{%
\section{Nuevo Notebook}\label{nuevo-notebook}}

    \hypertarget{apagar-la-aplicaciuxf3n-de-jupyter-notebooks}{%
\section{Apagar la aplicación de Jupyter
Notebooks}\label{apagar-la-aplicaciuxf3n-de-jupyter-notebooks}}

Al cerrar el navegador (o la pestaña) no se cerrará la aplicación
Jupyter Notebook. Para apagarlo completamente necesita cerrar el
terminal asociado.

Se pueden ejecutar muchas copias de la aplicación Jupyter Notebook y
aparecerán en una dirección similar. Sin embargo ya que con una sola
aplicación de Jupyter Notebooks se pueden abrir muchos ``documentos
notebook'', no se recomienda ejecutar varias copias de la aplicación.

    \hypertarget{cerrar-un-documento-notebook-cerrar-el-kernel-asociado}{%
\section{Cerrar un Documento Notebook (cerrar el Kernel
asociado)}\label{cerrar-un-documento-notebook-cerrar-el-kernel-asociado}}

Cuando se abre un Docuemento Notebook, su ``motor de cómputo'' (llamado
kernel) se inicia automáticamente. Al cerrar la pestaña del navegador,
no se apagará el kernel, en su lugar, el kernel seguirá funcionando
hasta que se apague explícitamente.

    Alternativamente, el ``dashboard'' estan señalados en verde los Notebook
que estan activos y se pueden cerrar desde ahi.

    \hypertarget{interfaz-de-usuario}{%
\section{Interfaz de Usuario}\label{interfaz-de-usuario}}

    \hypertarget{modos}{%
\section{Modos}\label{modos}}

\hypertarget{modo-de-ediciuxf3n}{%
\subsubsection{Modo de Edición}\label{modo-de-ediciuxf3n}}

El borde de la celda se encuentra en verde y se puede escribir
libremente, se entra este modo presionando ``Enter'' o con un clik en el
área de edición de alguna celda.

\hypertarget{modo-de-comandos}{%
\subsubsection{Modo de Comandos}\label{modo-de-comandos}}

El borde de la celda se encuentra en azul y permite el ingreso de
``atajos del teclado'' que permiten manipular las celdas de una manera
eficiente, se entra este modo presionando ``Esc'' o con un clik fuera
del área de edición de las celdas.

    \hypertarget{tipos-de-celdas}{%
\section{Tipos de Celdas}\label{tipos-de-celdas}}

\hypertarget{celdas-de-codigo-code-cells}{%
\subsubsection{Celdas de Codigo (Code
cells)}\label{celdas-de-codigo-code-cells}}

Permiten ejecutar editar y ejecutar codigo. El lenguaje depende del
Kernel que se haya seleccionado, el kernel por defecto es IPython que
corre codigo en Python. Cuando una celda de codigo se ejecuta, el codigo
que contiene la celda se manda al Kernel para ser procesado y para luego
mostrar el resultado como la salida de la celda. \#\#\# Celdas Markdown
(Markdown cells) Permite documentar el proceso de programación por medio
de texto ``Markdown'', que permite especificar texto en distintos
formatos, texto en negrita, cursivas, tablas, ecuaciones, html, imagenes
etc. \#\#\# Celdas crudas (Raw cells) Permite escribir texto que no sera
evaluado por el notebook

    Celdas Markdown

\hypertarget{encabezado-1}{%
\section{Encabezado 1}\label{encabezado-1}}

\hypertarget{encabezado-1.1}{%
\subsection{Encabezado 1.1}\label{encabezado-1.1}}

Negrita: \textbf{texto} o \textbf{texto}, cursivas: \emph{texto} or
\emph{texto} * viñeta 1 * viñeta 2 1. numeración 1 2. numeración 2

    Celdas Markdown

\hypertarget{tablas}{%
\section{Tablas}\label{tablas}}

\begin{longtable}[]{@{}lll@{}}
\toprule
Tiempo (s) & Posición X (cm) & Posicion Y (cm)\tabularnewline
\midrule
\endhead
1 & .45 & .2\tabularnewline
2 & .52 & .1\tabularnewline
3 & .71 & 0\tabularnewline
4 & .78 & -.05\tabularnewline
5 & .81 & -.07\tabularnewline
\bottomrule
\end{longtable}

    Celdas Markdown

\hypertarget{ecuaciones}{%
\section{Ecuaciones}\label{ecuaciones}}

Las Ecuaciones se ingresan por medio de el lenguaje \(\LaTeX\):

\begin{enumerate}
\def\labelenumi{\arabic{enumi}.}
\item
  \(\int_0^\infty x^ {-\alpha}\)
\item
  \(e^{i\pi} + 1 = 0\)
\item
  \(e^x=\sum_{i=0}^\infty \frac{1}{i!}x^i\)
\end{enumerate}

    Celdas Markdown

\hypertarget{hipervinculos-e-imagenes}{%
\section{Hipervinculos e Imagenes}\label{hipervinculos-e-imagenes}}

hipervinculo: \href{https://www.google.com}{Google}

    \hypertarget{celdas-de-codigo}{%
\section{Celdas de Codigo}\label{celdas-de-codigo}}

    \hypertarget{ejemplos-de-notebooks}{%
\section{Ejemplos de Notebooks}\label{ejemplos-de-notebooks}}

https://github.com/jupyter/jupyter/wiki/A-gallery-of-interesting-Jupyter-Notebooks

    \hypertarget{python}{%
\section{Python}\label{python}}

    Python es un lenguaje de programación de alto nivel, interpretado creado
por Guido van Rossum. La filosofía del lenguaje es darle prioridad a la
legibilidad de código.

Python consta de tipado dinámico y un manejo automático de memoria.
Soporta multiples paradigmas de programación, ya que soporta orientación
a objetos, programación imperativa, programación funcional y de
procedimientos.

\emph{Fuente:
\href{https://en.wikipedia.org/wiki/Python_(programming_language)}{Wikipedia}}

    \hypertarget{variables-en-python}{%
\section{Variables en Python}\label{variables-en-python}}

Las variable sirven para guardar datos y pueden cambiar su valor en
cualquier momento:

    \begin{Verbatim}[commandchars=\\\{\}]
{\color{incolor}In [{\color{incolor}1}]:} \PY{c+c1}{\PYZsh{} Los comentarios en python inician con un signo de numero (\PYZsh{})}
        \PY{n}{saludo} \PY{o}{=} \PY{l+s+s1}{\PYZsq{}}\PY{l+s+s1}{Hello world!}\PY{l+s+s1}{\PYZsq{}}
        \PY{n+nb}{print}\PY{p}{(}\PY{n}{saludo}\PY{p}{)}
\end{Verbatim}


    \begin{Verbatim}[commandchars=\\\{\}]
Hello world!

    \end{Verbatim}

    \begin{Verbatim}[commandchars=\\\{\}]
{\color{incolor}In [{\color{incolor}2}]:} \PY{c+c1}{\PYZsh{} Los valores de las variables pueden cambiar en cualquier momento}
        \PY{n}{x} \PY{o}{=} \PY{l+m+mi}{10}
        \PY{n}{y} \PY{o}{=} \PY{l+m+mi}{3}
        \PY{n}{y} \PY{o}{=} \PY{n}{x} \PY{o}{+} \PY{n}{y}
        \PY{n+nb}{print} \PY{p}{(}\PY{n}{y}\PY{p}{)}
\end{Verbatim}


    \begin{Verbatim}[commandchars=\\\{\}]
13

    \end{Verbatim}

    Variables en Python (continuación)

\hypertarget{reglas-para-nombrar-variables}{%
\section{Reglas para nombrar
variables}\label{reglas-para-nombrar-variables}}

\begin{itemize}
\tightlist
\item
  El nombre de una variable solo puede contener letras, números y
  guiones bajos y no puden empezar con un número.
\item
  No se permiten espacios
\item
  No se pueden utilizar palabras clave de
\end{itemize}

Ejemplos de nombres de variables correctas: temperatura, contador, x,
sensor\_1

Ejemplos de nombres de variables incorrectas: 12cont, cuarto 3, print,
def

    Variables en Python (continuación)

\begin{itemize}
\tightlist
\item
  Las variables enteras son variables numericas que \textbf{no
  contienen} un punto decimal.
\item
  Las variables de punto flotante son variables numericas que
  \textbf{contienen} un punto decimal.
\item
  Las variables de cadena de caracteres se encuentran entre comillas
  sencillas (') o comillas dobles (")
\end{itemize}

    \hypertarget{operaciones-aritmeticas}{%
\section{Operaciones aritmeticas}\label{operaciones-aritmeticas}}

las operaciones de suma, resta, multiplicacion, división, exponentes y
resto en division entera se representan por medio de los simbolos +, -,
*, /, **, \% respectivamente.

    \hypertarget{sentencias-de-control-condicionales}{%
\section{Sentencias de control
condicionales}\label{sentencias-de-control-condicionales}}

La sentencia if prueba una condición y luego responde a esa condición.
Si la condición es verdadera todas las sentencias del siguiente bloque
se ejecutarán. En caso contrario y en caso de haber sido incluida la
sentencia ``else'', el bloque de instrucciones inmediatamente despues se
ejecutarían.

    \begin{Verbatim}[commandchars=\\\{\}]
{\color{incolor}In [{\color{incolor}3}]:} \PY{n}{nombre1} \PY{o}{=} \PY{l+s+s1}{\PYZsq{}}\PY{l+s+s1}{Alfonso}\PY{l+s+s1}{\PYZsq{}}
        \PY{n}{nombre2} \PY{o}{=} \PY{l+s+s1}{\PYZsq{}}\PY{l+s+s1}{alfonso}\PY{l+s+s1}{\PYZsq{}}
        
        \PY{k}{if} \PY{n}{nombre1}\PY{o}{.}\PY{n}{lower}\PY{p}{(}\PY{p}{)} \PY{o}{==} \PY{n}{nombre2}\PY{o}{.}\PY{n}{lower}\PY{p}{(}\PY{p}{)}\PY{p}{:}
            \PY{n+nb}{print}\PY{p}{(}\PY{l+s+s1}{\PYZsq{}}\PY{l+s+s1}{Comparten el mismo nombre!}\PY{l+s+s1}{\PYZsq{}}\PY{p}{)}
        \PY{k}{else}\PY{p}{:}
            \PY{n+nb}{print}\PY{p}{(}\PY{l+s+s1}{\PYZsq{}}\PY{l+s+s1}{Las personas tienen distinto nombre}\PY{l+s+s1}{\PYZsq{}}\PY{p}{)}
\end{Verbatim}


    \begin{Verbatim}[commandchars=\\\{\}]
Comparten el mismo nombre!

    \end{Verbatim}

    Para verificar igualdad ``=='', inigualdad ``!='', mayor que
``\textgreater{}'', menor que ``\textgreater{}'', moyor o igual
``\textgreater{}='', menor o igual ``\textless{}='' y para verificar si
algun elemento forma parte de una lista ``in''.

    \hypertarget{sentencias-de-control-de-bucle}{%
\section{Sentencias de control de
bucle}\label{sentencias-de-control-de-bucle}}

\hypertarget{bucle-for}{%
\subsubsection{Bucle For}\label{bucle-for}}

Los bucles for son usados para iterar en los elementos de una secuencia.

    \begin{Verbatim}[commandchars=\\\{\}]
{\color{incolor}In [{\color{incolor}4}]:} \PY{k}{for} \PY{n}{x} \PY{o+ow}{in} \PY{n+nb}{range}\PY{p}{(}\PY{l+m+mi}{10}\PY{p}{)}\PY{p}{:}
            \PY{n+nb}{print} \PY{p}{(}\PY{n}{x}\PY{p}{)}
\end{Verbatim}


    \begin{Verbatim}[commandchars=\\\{\}]
0
1
2
3
4
5
6
7
8
9

    \end{Verbatim}

    \begin{Verbatim}[commandchars=\\\{\}]
{\color{incolor}In [{\color{incolor}5}]:} \PY{c+c1}{\PYZsh{} La estructura de datos de este ejemplo es una lista literal}
        \PY{k}{for} \PY{n}{nombre} \PY{o+ow}{in} \PY{p}{[}\PY{l+s+s1}{\PYZsq{}}\PY{l+s+s1}{Luis}\PY{l+s+s1}{\PYZsq{}}\PY{p}{,} \PY{l+s+s1}{\PYZsq{}}\PY{l+s+s1}{Pedro}\PY{l+s+s1}{\PYZsq{}}\PY{p}{,} \PY{l+s+s1}{\PYZsq{}}\PY{l+s+s1}{Margarita}\PY{l+s+s1}{\PYZsq{}}\PY{p}{,} \PY{l+s+s1}{\PYZsq{}}\PY{l+s+s1}{Alfonso}\PY{l+s+s1}{\PYZsq{}}\PY{p}{]}\PY{p}{:}
            \PY{n+nb}{print} \PY{p}{(}\PY{n}{nombre}\PY{p}{)}
\end{Verbatim}


    \begin{Verbatim}[commandchars=\\\{\}]
Luis
Pedro
Margarita
Alfonso

    \end{Verbatim}

    Sentencias de control de bucle (continuación)

\hypertarget{bucle-while}{%
\subsubsection{Bucle While}\label{bucle-while}}

Los bucles While verifican una condición y en caso de ser verdadera
ejecutan el bloque de codigo subsecuente, este se mantiene en ejecucion
mientras la condición pemanezca verdadera:

    \begin{Verbatim}[commandchars=\\\{\}]
{\color{incolor}In [{\color{incolor}6}]:} \PY{n}{temperatura} \PY{o}{=} \PY{l+m+mi}{25}
        \PY{k}{while} \PY{n}{temperatura} \PY{o}{\PYZlt{}} \PY{l+m+mi}{32}\PY{p}{:}
            \PY{n+nb}{print} \PY{p}{(}\PY{l+s+s1}{\PYZsq{}}\PY{l+s+s1}{Temperatura =}\PY{l+s+s1}{\PYZsq{}} \PY{o}{+} \PY{n+nb}{str}\PY{p}{(}\PY{n}{temperatura}\PY{p}{)} \PY{o}{+} \PY{l+s+s1}{\PYZsq{}}\PY{l+s+s1}{ el clima es agradable}\PY{l+s+s1}{\PYZsq{}}\PY{p}{)}
            \PY{n}{temperatura}  \PY{o}{=} \PY{n}{temperatura} \PY{o}{+} \PY{l+m+mi}{1}
        
        \PY{n+nb}{print}\PY{p}{(}\PY{l+s+s1}{\PYZsq{}}\PY{l+s+s1}{Ahora el clima es caluroso!!!}\PY{l+s+s1}{\PYZsq{}}\PY{p}{)}
\end{Verbatim}


    \begin{Verbatim}[commandchars=\\\{\}]
Temperatura =25 el clima es agradable
Temperatura =26 el clima es agradable
Temperatura =27 el clima es agradable
Temperatura =28 el clima es agradable
Temperatura =29 el clima es agradable
Temperatura =30 el clima es agradable
Temperatura =31 el clima es agradable
Ahora el clima es caluroso!!!

    \end{Verbatim}

    \begin{Verbatim}[commandchars=\\\{\}]
{\color{incolor}In [{\color{incolor}7}]:} \PY{n}{password} \PY{o}{=} \PY{l+s+s1}{\PYZsq{}}\PY{l+s+s1}{\PYZsq{}}
        \PY{k}{while} \PY{n}{password} \PY{o}{!=} \PY{l+s+s1}{\PYZsq{}}\PY{l+s+s1}{****}\PY{l+s+s1}{\PYZsq{}}\PY{p}{:}
            \PY{n}{password} \PY{o}{=} \PY{n+nb}{input}\PY{p}{(}\PY{l+s+s1}{\PYZsq{}}\PY{l+s+s1}{Favor de ingresar el password}\PY{l+s+s1}{\PYZsq{}}\PY{p}{)}
        
        \PY{n+nb}{print}\PY{p}{(}\PY{l+s+s1}{\PYZsq{}}\PY{l+s+s1}{password correcto, bienvenido!}\PY{l+s+s1}{\PYZsq{}}\PY{p}{)}    
\end{Verbatim}


    \begin{Verbatim}[commandchars=\\\{\}]
Favor de ingresar el passwordhola
Favor de ingresar el passwordtest
Favor de ingresar el password****
password correcto, bienvenido!

    \end{Verbatim}

    \hypertarget{listas}{%
\section{Listas}\label{listas}}

Una lista es una colección de elementos, que se almacena en una
variable.

    \begin{Verbatim}[commandchars=\\\{\}]
{\color{incolor}In [{\color{incolor}8}]:} \PY{n}{misMascotas} \PY{o}{=} \PY{p}{[}\PY{l+s+s1}{\PYZsq{}}\PY{l+s+s1}{pinky}\PY{l+s+s1}{\PYZsq{}}\PY{p}{,} \PY{l+s+s1}{\PYZsq{}}\PY{l+s+s1}{mariposa}\PY{l+s+s1}{\PYZsq{}}\PY{p}{,} \PY{l+s+s1}{\PYZsq{}}\PY{l+s+s1}{tohui}\PY{l+s+s1}{\PYZsq{}}\PY{p}{,} \PY{l+s+s1}{\PYZsq{}}\PY{l+s+s1}{sheldon}\PY{l+s+s1}{\PYZsq{}}\PY{p}{]}
        \PY{n+nb}{print}\PY{p}{(}\PY{n}{misMascotas}\PY{p}{[}\PY{l+m+mi}{0}\PY{p}{]} \PY{o}{+} \PY{l+s+s1}{\PYZsq{}}\PY{l+s+s1}{,}\PY{l+s+s1}{\PYZsq{}} \PY{o}{+} \PY{n}{misMascotas}\PY{p}{[}\PY{l+m+mi}{2}\PY{p}{]}\PY{p}{)}
\end{Verbatim}


    \begin{Verbatim}[commandchars=\\\{\}]
pinky,tohui

    \end{Verbatim}

    \begin{Verbatim}[commandchars=\\\{\}]
{\color{incolor}In [{\color{incolor}9}]:} \PY{c+c1}{\PYZsh{} Para obtener el ultimo miebro de la lista se puede utilizar indices negativos:}
        \PY{n+nb}{print}\PY{p}{(}\PY{n}{misMascotas}\PY{p}{[}\PY{o}{\PYZhy{}}\PY{l+m+mi}{1}\PY{p}{]} \PY{o}{+} \PY{l+s+s1}{\PYZsq{}}\PY{l+s+s1}{,}\PY{l+s+s1}{\PYZsq{}} \PY{o}{+} \PY{n}{misMascotas}\PY{p}{[}\PY{o}{\PYZhy{}}\PY{l+m+mi}{4}\PY{p}{]}\PY{p}{)}
\end{Verbatim}


    \begin{Verbatim}[commandchars=\\\{\}]
sheldon,pinky

    \end{Verbatim}

    \begin{Verbatim}[commandchars=\\\{\}]
{\color{incolor}In [{\color{incolor}10}]:} \PY{c+c1}{\PYZsh{} Para saber el indice de un elemento en la lista se usa la funcion index()}
         \PY{n}{misMascotas}\PY{o}{.}\PY{n}{index}\PY{p}{(}\PY{l+s+s1}{\PYZsq{}}\PY{l+s+s1}{tohui}\PY{l+s+s1}{\PYZsq{}}\PY{p}{)}
\end{Verbatim}


\begin{Verbatim}[commandchars=\\\{\}]
{\color{outcolor}Out[{\color{outcolor}10}]:} 2
\end{Verbatim}
            
    \begin{Verbatim}[commandchars=\\\{\}]
{\color{incolor}In [{\color{incolor}11}]:} \PY{c+c1}{\PYZsh{} Para saber si una lista contiene un elemento se utiliza la palabra clave \PYZdq{}in\PYZdq{}}
         \PY{n+nb}{print}\PY{p}{(}\PY{l+s+s2}{\PYZdq{}}\PY{l+s+s2}{firulais}\PY{l+s+s2}{\PYZdq{}} \PY{o+ow}{in} \PY{n}{misMascotas}\PY{p}{)}
\end{Verbatim}


    \begin{Verbatim}[commandchars=\\\{\}]
False

    \end{Verbatim}

    Listas (continuación)

    \begin{Verbatim}[commandchars=\\\{\}]
{\color{incolor}In [{\color{incolor}12}]:} \PY{c+c1}{\PYZsh{} Agregar elementos al final de una lista}
         \PY{c+c1}{\PYZsh{} lista original: misMascotas = [\PYZsq{}pinky\PYZsq{}, \PYZsq{}mariposa\PYZsq{}, \PYZsq{}tohui\PYZsq{}, \PYZsq{}sheldon\PYZsq{}]}
         \PY{n}{misMascotas}\PY{o}{.}\PY{n}{append}\PY{p}{(}\PY{l+s+s1}{\PYZsq{}}\PY{l+s+s1}{capitan}\PY{l+s+s1}{\PYZsq{}}\PY{p}{)}
         \PY{n+nb}{print}\PY{p}{(}\PY{n}{misMascotas}\PY{p}{)}
\end{Verbatim}


    \begin{Verbatim}[commandchars=\\\{\}]
['pinky', 'mariposa', 'tohui', 'sheldon', 'capitan']

    \end{Verbatim}

    \begin{Verbatim}[commandchars=\\\{\}]
{\color{incolor}In [{\color{incolor}13}]:} \PY{c+c1}{\PYZsh{} Agregar elementos en un indice determinado de una lista}
         \PY{n}{misMascotas}\PY{o}{.}\PY{n}{insert}\PY{p}{(}\PY{l+m+mi}{1}\PY{p}{,} \PY{l+s+s1}{\PYZsq{}}\PY{l+s+s1}{bruno}\PY{l+s+s1}{\PYZsq{}}\PY{p}{)}
         \PY{n+nb}{print}\PY{p}{(}\PY{n}{misMascotas}\PY{p}{)}
\end{Verbatim}


    \begin{Verbatim}[commandchars=\\\{\}]
['pinky', 'bruno', 'mariposa', 'tohui', 'sheldon', 'capitan']

    \end{Verbatim}

    \begin{Verbatim}[commandchars=\\\{\}]
{\color{incolor}In [{\color{incolor}14}]:} \PY{c+c1}{\PYZsh{} Para ordenar alfabeticamente una lista}
         \PY{n}{misMascotas}\PY{o}{.}\PY{n}{sort}\PY{p}{(}\PY{p}{)}
         \PY{n+nb}{print}\PY{p}{(}\PY{n}{misMascotas}\PY{p}{)}
\end{Verbatim}


    \begin{Verbatim}[commandchars=\\\{\}]
['bruno', 'capitan', 'mariposa', 'pinky', 'sheldon', 'tohui']

    \end{Verbatim}

    \begin{Verbatim}[commandchars=\\\{\}]
{\color{incolor}In [{\color{incolor}15}]:} \PY{c+c1}{\PYZsh{} Para saber el número de elementos en una lista}
         \PY{n+nb}{print}\PY{p}{(}\PY{n+nb}{len}\PY{p}{(}\PY{n}{misMascotas}\PY{p}{)}\PY{p}{)}
\end{Verbatim}


    \begin{Verbatim}[commandchars=\\\{\}]
6

    \end{Verbatim}

    Listas (continuación)

    \begin{Verbatim}[commandchars=\\\{\}]
{\color{incolor}In [{\color{incolor}16}]:} \PY{c+c1}{\PYZsh{} Eliminar elementos de una lista en un indice determinado}
         \PY{c+c1}{\PYZsh{} Contenido Actual de misMascotas = [\PYZsq{}bruno\PYZsq{}, \PYZsq{}capitan\PYZsq{}, \PYZsq{}mariposa\PYZsq{}, \PYZsq{}pinky\PYZsq{}, \PYZsq{}sheldon\PYZsq{}, \PYZsq{}tohui\PYZsq{}]}
         \PY{k}{del} \PY{n}{misMascotas}\PY{p}{[}\PY{l+m+mi}{0}\PY{p}{]}
         \PY{n+nb}{print}\PY{p}{(}\PY{n}{misMascotas}\PY{p}{)}
\end{Verbatim}


    \begin{Verbatim}[commandchars=\\\{\}]
['capitan', 'mariposa', 'pinky', 'sheldon', 'tohui']

    \end{Verbatim}

    \begin{Verbatim}[commandchars=\\\{\}]
{\color{incolor}In [{\color{incolor}17}]:} \PY{c+c1}{\PYZsh{} Eliminar elementos de una lista por valor}
         \PY{n}{misMascotas}\PY{o}{.}\PY{n}{remove}\PY{p}{(}\PY{l+s+s1}{\PYZsq{}}\PY{l+s+s1}{mariposa}\PY{l+s+s1}{\PYZsq{}}\PY{p}{)}
         \PY{n+nb}{print}\PY{p}{(}\PY{n}{misMascotas}\PY{p}{)}
\end{Verbatim}


    \begin{Verbatim}[commandchars=\\\{\}]
['capitan', 'pinky', 'sheldon', 'tohui']

    \end{Verbatim}

    \begin{Verbatim}[commandchars=\\\{\}]
{\color{incolor}In [{\color{incolor}18}]:} \PY{c+c1}{\PYZsh{} Para obtener el ultimo elemento de una lista y eliminarlo al mimso tiempo}
         \PY{n}{mascota} \PY{o}{=} \PY{n}{misMascotas}\PY{o}{.}\PY{n}{pop}\PY{p}{(}\PY{p}{)}
         \PY{n+nb}{print}\PY{p}{(}\PY{n}{misMascotas}\PY{p}{)}
         \PY{n+nb}{print}\PY{p}{(}\PY{n}{mascota}\PY{p}{)}
\end{Verbatim}


    \begin{Verbatim}[commandchars=\\\{\}]
['capitan', 'pinky', 'sheldon']
tohui

    \end{Verbatim}

    \begin{Verbatim}[commandchars=\\\{\}]
{\color{incolor}In [{\color{incolor}19}]:} \PY{c+c1}{\PYZsh{} Para obtener cualquier elemento de una lista y eliminarlo al mimso tiempo}
         \PY{n}{mascota} \PY{o}{=} \PY{n}{misMascotas}\PY{o}{.}\PY{n}{pop}\PY{p}{(}\PY{l+m+mi}{0}\PY{p}{)}
         \PY{n+nb}{print}\PY{p}{(}\PY{n}{misMascotas}\PY{p}{)}
         \PY{n+nb}{print}\PY{p}{(}\PY{n}{mascota}\PY{p}{)}
\end{Verbatim}


    \begin{Verbatim}[commandchars=\\\{\}]
['pinky', 'sheldon']
capitan

    \end{Verbatim}

    \hypertarget{funciones}{%
\section{Funciones}\label{funciones}}

Una función es un bloque de código que es ejecutado cuando se hace una
llamada a la función. La función puede recibir valores llamados
parametros y tambien puede regresar datos como resultado

    \begin{Verbatim}[commandchars=\\\{\}]
{\color{incolor}In [{\color{incolor}20}]:} \PY{k}{def} \PY{n+nf}{sum}\PY{p}{(}\PY{n}{a}\PY{p}{,}\PY{n}{b}\PY{p}{)}\PY{p}{:}
             \PY{k}{return} \PY{n}{a} \PY{o}{+} \PY{n}{b}
         
         \PY{n+nb}{print}\PY{p}{(}\PY{n+nb}{sum}\PY{p}{(}\PY{l+m+mi}{23}\PY{p}{,}\PY{l+m+mi}{24}\PY{p}{)}\PY{p}{)}
\end{Verbatim}


    \begin{Verbatim}[commandchars=\\\{\}]
47

    \end{Verbatim}

    \hypertarget{vistazo-a-tuxf3picos-avanzados}{%
\section{Vistazo a Tópicos
Avanzados}\label{vistazo-a-tuxf3picos-avanzados}}

 

    Algunas de las librerias y paqueterias que conforman la pila científica
básica de Python son:

\begin{itemize}
\tightlist
\item
  NumPy para manipulación básica de matrices
\item
  SciPy, computación científica en Python, incluyendo procesamiento y
  optimización de señales
\item
  Matplotlib visualización y graficación.
\item
  Pandas para manipulación de datos
\item
  Scikit-learn, aprendizaje de maquina (machine learning)
\end{itemize}


    % Add a bibliography block to the postdoc
    
    
    
    \end{document}
